%% Generated by Sphinx.
\def\sphinxdocclass{jupyterBook}
\documentclass[letterpaper,10pt,english]{jupyterBook}
\ifdefined\pdfpxdimen
   \let\sphinxpxdimen\pdfpxdimen\else\newdimen\sphinxpxdimen
\fi \sphinxpxdimen=.75bp\relax
\ifdefined\pdfimageresolution
    \pdfimageresolution= \numexpr \dimexpr1in\relax/\sphinxpxdimen\relax
\fi
%% let collapsible pdf bookmarks panel have high depth per default
\PassOptionsToPackage{bookmarksdepth=5}{hyperref}
%% turn off hyperref patch of \index as sphinx.xdy xindy module takes care of
%% suitable \hyperpage mark-up, working around hyperref-xindy incompatibility
\PassOptionsToPackage{hyperindex=false}{hyperref}
%% memoir class requires extra handling
\makeatletter\@ifclassloaded{memoir}
{\ifdefined\memhyperindexfalse\memhyperindexfalse\fi}{}\makeatother

\PassOptionsToPackage{booktabs}{sphinx}
\PassOptionsToPackage{colorrows}{sphinx}

\PassOptionsToPackage{warn}{textcomp}

\catcode`^^^^00a0\active\protected\def^^^^00a0{\leavevmode\nobreak\ }
\usepackage{cmap}
\usepackage{fontspec}
\defaultfontfeatures[\rmfamily,\sffamily,\ttfamily]{}
\usepackage{amsmath,amssymb,amstext}
\usepackage{polyglossia}
\setmainlanguage{english}



\setmainfont{FreeSerif}[
  Extension      = .otf,
  UprightFont    = *,
  ItalicFont     = *Italic,
  BoldFont       = *Bold,
  BoldItalicFont = *BoldItalic
]
\setsansfont{FreeSans}[
  Extension      = .otf,
  UprightFont    = *,
  ItalicFont     = *Oblique,
  BoldFont       = *Bold,
  BoldItalicFont = *BoldOblique,
]
\setmonofont{FreeMono}[
  Extension      = .otf,
  UprightFont    = *,
  ItalicFont     = *Oblique,
  BoldFont       = *Bold,
  BoldItalicFont = *BoldOblique,
]



\usepackage[Bjarne]{fncychap}
\usepackage[,numfigreset=1,mathnumfig]{sphinx}

\fvset{fontsize=\small}
\usepackage{geometry}


% Include hyperref last.
\usepackage{hyperref}
% Fix anchor placement for figures with captions.
\usepackage{hypcap}% it must be loaded after hyperref.
% Set up styles of URL: it should be placed after hyperref.
\urlstyle{same}


\usepackage{sphinxmessages}



        % Start of preamble defined in sphinx-jupyterbook-latex %
         \usepackage[Latin,Greek]{ucharclasses}
        \usepackage{unicode-math}
        % fixing title of the toc
        \addto\captionsenglish{\renewcommand{\contentsname}{Contents}}
        \hypersetup{
            pdfencoding=auto,
            psdextra
        }
        % End of preamble defined in sphinx-jupyterbook-latex %
        

\title{My sample book}
\date{Mar 31, 2024}
\release{}
\author{The Jupyter Book Community}
\newcommand{\sphinxlogo}{\vbox{}}
\renewcommand{\releasename}{}
\makeindex
\begin{document}

\pagestyle{empty}
\sphinxmaketitle
\pagestyle{plain}
\sphinxtableofcontents
\pagestyle{normal}
\phantomsection\label{\detokenize{intro::doc}}


\sphinxAtStartPar
Introduction
\begin{itemize}
\item {} 
\sphinxAtStartPar
{\hyperref[\detokenize{part1::doc}]{\sphinxcrossref{Part 1}}}

\item {} 
\sphinxAtStartPar
{\hyperref[\detokenize{part2::doc}]{\sphinxcrossref{Part 2}}}

\item {} 
\sphinxAtStartPar
{\hyperref[\detokenize{part3::doc}]{\sphinxcrossref{Part 3}}}

\end{itemize}

\sphinxstepscope


\chapter{Part 1}
\label{\detokenize{part1:part-1}}\label{\detokenize{part1::doc}}

\section{Humanal Digitanities}
\label{\detokenize{part1:humanal-digitanities}}
\sphinxAtStartPar
\sphinxincludegraphics{{hd-1}.png}

\sphinxAtStartPar
It started with oversharing in a group chat. (And is that not the purest distillation of what goes on in a writing workshop, where a handful of my work first saw the light of day?)

\sphinxAtStartPar
Specifically, I had Wayback Machined myself to locate some old writing and uncovered a post — inspired by a \sphinxhref{https://www.mcsweeneys.net/articles/final-sentences-of-essays-i-wrote-in-college}{McSweeney’s piece} — where I shared the final sentences of papers I’d written for school. And just to be very clear right away, the whole reason this project can happen is because I keep everything.

\sphinxAtStartPar
Back to the group chat, which I occupy with a small group of friends from my library science grad program. In the spirit of embarrassing myself to people I trust, I shared some of the funniest or weirdest final sentences with them. Like this one:

\sphinxAtStartPar
\sphinxincludegraphics{{hd-2}.png}

\sphinxAtStartPar
But then came the epiphany: I have all this stuff, so why not use it somehow? And maybe especially in a way that fulfills the requirements of my current digital humanities independent study?

\sphinxAtStartPar
\sphinxincludegraphics{{hd-3}.png}

\sphinxAtStartPar
This could be a section on its own — and maybe it will be someday — but I cannot express how much I truly thought “digital memoir” was already a thing in DH. So many great projects analyze famous people’s writings or relationships through a DH lens that I just figured that someone had probably already turned the camera around and examined their own works in this manner. I thought about it more, though, and realized there’s not much external value in pouring time and resources into a personal project in those spaces that have time and resources to analyze, like, Francis Bacon’s affairs. I’ve seen pieces like \sphinxhref{https://www.nytimes.com/interactive/2019/12/27/opinion/sunday/decade-google-search.html}{“My Decade in Google Searches”} and \sphinxhref{https://www.nytimes.com/2019/01/04/style/modern-love-end-of-marriage-google-maps.html}{“Tracking the Demise of My Marriage on Google Maps”} that extract (or just represent) the relationship between technology and personality. But in spite of their emotional appeal, these are typically brief analyses, initial reactions to the tool itself or a representation of how a tool can appear “humanlike.” But since my independent study is self\sphinxhyphen{}guided to the point that I can declare what I’d like to get out of it and then just…define “digital memoir” for myself? Hello!

\sphinxAtStartPar
So here’s the very meta, very basic version of the “digital memoir” proposal I came up with: I’m using digital tools to explore and analyze my own collection of memoir\sphinxhyphen{}ish works in hopes that I learn something about… well, myself. Or how I choose to write about myself. Or how I interact with myself through words. Or, to put it in formal academic terms, I am putting the “me” back in “memoir.”


\section{The Corpus}
\label{\detokenize{part1:the-corpus}}
\sphinxAtStartPar
This is a collection of works I wrote that I feel constitute “memoir.” I’ve done a great deal of personal writing, but when it came to zeroing in on the right feel for this corpus, for this project, I found the most important way I eliminated potential works was if they felt too blog\sphinxhyphen{}like. To put it in insultingly simple terms, a memoir’s takeaways should slow\sphinxhyphen{}burn the audience, rather than be wrapped up neatly at the end as a blog post’s call to action might. “Is this a slow burn or am I holding the audience’s hand?” was a useful shorthand version that cut my corpus down from 50\sphinxhyphen{}ish pieces to 37.

\sphinxAtStartPar
The 37 pieces are broadly broken up as follows:
\begin{itemize}
\item {} 
\sphinxAtStartPar
24 were self\sphinxhyphen{}published, mostly in my personal newsletter

\item {} 
\sphinxAtStartPar
12 were school assignments

\item {} 
\sphinxAtStartPar
1 was written for a magazine

\end{itemize}

\sphinxAtStartPar
So that was that. I’ve written a lot about myself. It has always made me feel a little bit weird to have shared so much about myself, but at the same time, I appreciate that I have this much content I can look at again in a new way. (Sidebar: I recently read an article about how momfluencers justify sharing their kids’ lives online before they can consent by saying “They’ll be glad someday that they can look back on all this.” And my first thought was, “But why does that necessitate putting it online when you can just get a giant external hard drive?” And then I remembered this project and went, welp, I guess I’m glad I’m only exploiting myself.)


\section{A Quick Tangent About Basketball}
\label{\detokenize{part1:a-quick-tangent-about-basketball}}
\sphinxAtStartPar
I was a journalist in a past life, and specifically a basketball journalist, which meant I spent a lot of time trying to make sense of statistics. Which stats make a basketball player “good”? Lots of people will say scoring a lot of points does — it directly helps your team outscore the other team, after all. And if you can throw in a consistently high number of rebounds, assists, or both, all the better! There are also some popular ways to combine certain stats to tell a story: the assist\sphinxhyphen{}to\sphinxhyphen{}turnover ratio allegedly shows how good a point guard is at taking care of the basketball, while “stocks” (steals + blocks) demonstrate a player’s defensive prowess. But it’s also pretty widely acknowledged that stats alone don’t indicate a player is “good” — there are certain intangibles, we’d say, that make a player an asset to their team. Leadership, maturity, being the last one in the gym every night, aspects of their person that make a player stand out even if their stats don’t necessarily jump off the page.


\section{Research Questions}
\label{\detokenize{part1:research-questions}}
\sphinxAtStartPar
Finally, here’s what I wanted to know:
\begin{itemize}
\item {} 
\sphinxAtStartPar
Did my writing get better (or worse) over time?
\begin{itemize}
\item {} 
\sphinxAtStartPar
What do “better” and “worse” mean?

\end{itemize}

\item {} 
\sphinxAtStartPar
Can I identify any idiosyncrasies that changed (or stayed the same) over time?

\item {} 
\sphinxAtStartPar
What differences exist between the mediums of publication — e.g., a piece I wrote knowing others would see it vs. a piece I wrote for an instructor — if any?

\end{itemize}

\sphinxAtStartPar
And while I found it valuable to look at the corpus as a whole for personal enjoyment (e.g., when I topic modeled and went, “ah, yes, these topics do make sense as descriptors of the things I write about a lot”), I knew that any further distant reading would benefit from a closer look. Yes, there’s something here — this player is “good”; this piece “is especially readable” or “has short sentences” — but I needed to divide my work into smaller pieces to understand what exactly was going on.

\sphinxstepscope


\chapter{Part 2}
\label{\detokenize{part2:part-2}}\label{\detokenize{part2::doc}}
\sphinxAtStartPar
Part 2

\sphinxstepscope


\chapter{Part 3}
\label{\detokenize{part3:part-3}}\label{\detokenize{part3::doc}}
\sphinxAtStartPar
Part 3







\renewcommand{\indexname}{Index}
\printindex
\end{document}